
% Einleitung

Die vorliegende Arbeit wurde im Rahmen eines Auftrags des Leistungs- und Qualit\"atsmanagements (LQM) der
Universität Graz verfasst. Sie hat auf der Vorarbeit des LQM aufgebaut und sollte diese weiterf\"uhren
und vertiefen. \\

Alle drei Jahre muss eine \"osterreichische Universität dem Bundesministerium f\"ur Bildung und Forschung
bekanntgeben wie viele \textit{pr\"ufungsaktive} Studierende sie voraussichtlich in \textbf{drei Jahren}
in der Zukunft haben wird. Umso genauer eine Universität diese Zahl vorhersagen kann, desto besser.
Wenn die jeweilige Universität zu wenig pr\"ufungsaktive Studierende vorhersagt, bekommt sie vom Bundesministerium
weniger Budget zugesprochen und kann somit den Universitätsbetrieb nicht optimal finanzieren. Auf der anderen Seite,
wenn die jeweilige Universität zu viel pr\"ufungsaktive Studierende vorhersagt, bekommt sie zuerst mehr Budget zugesprochen,
muss dieses aber sp\"ater wieder zur\"uckzahlen. Das kann dann bei gr\"o{\ss}eren geplanten Projekten, wie beispielsweise Laboren,
zum Abbruch des Projektes f\"uhren und somit einen Schaden verursachen. \\

\noindent Generell gilt eine studierende Person in einem Jahr als pr\"ufungsaktiv, falls sie:

\begin{itemize}
    \item 16 oder mehr ECTS in diesem Jahr positiv absolviert hat, oder sie
    \item 8 oder mehr Semesterwochenstunden positiv absolviert hat, oder sie
    \item in diesem Studienjahr ihr Studium positiv abschlie{\ss}t.
\end{itemize}
Andernfalls gilt die Person als \textit{nicht} pr\"ufungsaktiv. \\



% bisherige Herangehensweise

\section{Bisherige Herangehensweise}

Vom LQM wurde mittels unterschiedlicher Machine Learning Ans\"atze eine Sch\"atzung der Pr\"ufungsaktivit\"at
f\"ur das kommende Jahr auf der Ebene der Studierenden gemacht. 
Das bedeutet, es wurde versucht f\"ur jeden Studierenden vorherzusagen, ob er im darauffolgenden Jahr pr\"ufungsaktiv sein wird oder nicht.
Diese Modelle wurden eingesetzt, um Merkmalskombinationen von Studierenden herauszufinden,
die auf eine hohe Wahrscheinlichkeit der Pr\"ufungsaktivit\"at schlie{\ss}en lassen.
Weiters wurden Ans\"atze auf aggregierter Ebene \"uberlegt. Das bedeutet, dass man nicht die Pr\"ufungsaktivit\"at jedes
einzelnen Studierenden vorhersagen will. Stattdessen
versucht man den Anteil an einer Menge von Studierenden, die pr\"ufungsaktiv sein werden, vorherzusagen.
Die Ans\"atze auf aggregierter Ebene sind bisher noch nicht vollst\"andig formuliert und ausprobiert worden. \\


Bisher wurde das Problem indirekt bearbeitet, indem man probiert hat, die Pr\"ufungsaktivit\"at
immer f\"ur das folgende Jahr, vorherzusagen. Man hat keine Vorhersagen in die weitere Zukunft gemacht.
Man hat versucht die Pr\"ufungsaktivit\"at anhand der Merkmale von Studierenden zu erkl\"aren. \\

