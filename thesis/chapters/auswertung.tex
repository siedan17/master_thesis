

\section{Auswertung}
\label{sec:auswertung}
Die Auswertung kann in drei Kategorien unterteilt werden. Erstens werden die verschiedenen Machine Learning Modelle bewertet.
Diese werden in \hyperref[sec:appr1]{Ansatz 1} f\"ur P1 zur Regression, und in \hyperref[sec:appr2]{Ansatz 3} f\"ur P1 zur Klassifikation,
beziehungsweise Sch\"atzung von Wahrscheinlichkeiten, verwendet.
Zweitens werden die \"ubergeordneten Ans\"atze f\"ur P1 an sich bewertet.
Drittens versucht man die beiden Ans\"atze f\"ur P2 zu bewerten. \\

Generell werden die Machine Learning Modelle immer f\"ur Studierende, die sich in ihrem ersten Studienjahr und f\"ur Studierende, die sich im
zweiten oder h\"oheren Studienjahr befinden, gebildet. Diese Unterscheidung wird durchgef\"uhrt, da man f\"ur Studierende in ihrem ersten Jahr noch
weniger Merkmale zur Verf\"ugung hat. Beispielsweise hat man keine Daten \"uber bereits absolvierte ECTS. Demnach wird in der Auswertung
auch immer in diese beiden Gruppen unterteilt.


\subsection{Machine Learning Modelle}
In Ansatz 1 f\"ur P1 handelt es sich um ein Regressionsproblem. Man versucht anhand von Eigenschaften einer studierenden Person am Beginn eines Studienjahres
die erreichte ECTS Anzahl am Ende dieses Jahres vorherzusagen. Es werden erstens der
\textit{Root Mean Squared Error}:

$$ \operatorname{RMSE} = \sqrt{\frac{1}{n}\sum_{i = 1}^{n}(h(\mathbf{x}_i)-y_i)^2} $$

(wobei n die Anzahl der Testbeispiele ist), zweitens der \textit{Mean Absolute Error}:

$$ \operatorname{MAE} = \frac{1}{n}\sum_{i = 1}^{n}|h(\mathbf{x}_i) - y_i| $$

und drittens der \textit{Coefficient of determination $R^2$} berechnet:

$$ R^2 = 1 - \frac{\sum_{i = 1}^{n}(h(\mathbf{x}_i) - y_i)^2}{\sum_{i=1}^n(\bar{y}-y_i)^2} $$

Die ersten beiden bilden, je nach Entscheidung, ob man den bedingten Erwartungswert, oder den bedingten Median
approximieren m\"ochte, eine Sch\"atzung des empirischen Risikos. Die Modelle werden anhand dieser Metrik trainiert und anschlie{\ss}end
auch mit Daten, die man im Training nicht verwendet hat, getestet. Der Coefficient of determination gibt an, wie gut das Modell
die Variabilit\"at der tats\"achlichen Werte (im Bezug zum Mittelwert) beschreiben kann. Der Wert liegt normalerweise zwischen 0 und 1. Sollte der Wert negativ sein,
bedeutet dies, dass die Sch\"atzung des Modells schlechter die Zielwerte approximieren kann, als das arithmetische Mittel der Zielwerte \cite{guttag}.  \\


Bei Ansatz 3 handelt es sich um Klassifikationsmodell. Entweder ist eine studierende Person am Ende eines Studienjahres pr\"ufungsaktiv oder nicht. Das soll
durch Eigenschaften dieser Person, die man am Beginn des Studienjahres zur Verf\"ugung hat, vorhergesagt werden. Jedes Modell liefert f\"ur jede studierende Person
einen Wert zwischen 0 und 1, je nachdem wie wahrscheinlich es ist, dass die studierende Person pr\"ufungsaktiv ist oder nicht. Erst anhand eines manuell festgelegten
Schwellwertes klassifiziert das Modell in die jeweilige Klasse. \\

Man hat f\"ur die Auswertung aller Modelle eine \textit{Confusion Matrix} gebildet, die folgende Form besitzt:

$$ \left[ \begin{array}{rr} \text{True Negatives} & \text{False Positives}  \\  \text{False Negatives} & \text{True Positives} \\  \end{array}\right] .$$

Diese wird wiederum anhand von Testdaten gebildet, die der Algorithmus w\"ahrend des Trainings nicht gesehen hat. Weiters wird die \textit{Accuracy} anhand dieser Daten berechnet:

$$ Accuracy = \frac{TP + TN}{TP + TN + FP + FN}.$$

Es wird jedoch aufgrund der Problemstellung nur die zusammengefasste Anzahl an pr\"ufungsaktiven Studierenden ben\"otigt. Deswegen wird keine exakte Klassifizierung durchgef\"uhrt,
sondern es wird pro studierender Person ein Wert zwischen 0 und 1 ausgegeben, je nachdem wie wahrscheinlich es ist, dass diese Person pr\"ufungsaktiv sein wird oder nicht.
Mit diesen Werten wird die Anzahl der erwarteten pr\"ufungsaktiven Studierenden berechnet.
Diese erwartete Anzahl wird mit der tats\"achlichen Anzahl an pr\"ufungsaktiven Studierenden verglichen.
Vor allem diese Differenz ist f\"ur die Wahl des besten Vorhersagemodells ausschlaggebend. Je kleiner die Differenz ist, desto besser ist das Modell.



\subsection{Ans\"atze f\"ur P1}
Bei Ansatz 1 wird \"uberpr\"uft, ob es Sinn macht die erreichten ECTS der Studiereden Jahr f\"ur Jahr zu sch\"atzen. Sollten die Sch\"atzungen der
ECTS Werte geringe RMSE und MAE im Bereich bis 5 ECTS ergeben, wird eine Vorhersage Jahr f\"ur Jahr durchgef\"uhrt.
Anschlie{\ss}end wird im letzten Jahr, f\"ur das die ECTS gesch\"atzt wurden, entschieden, ob die studierende Person pr\"ufungsaktiv war oder nicht.
Diese verkettete Sch\"atzung findet auch auf einem eigenen Testdatensatz statt. Nun wird bewertet, wie weit sich die gesch\"atzte Anzahl an pr\"ufungsaktvien
Studierenden von der tats\"achlichen Anzahl in diesem Jahr unterscheidet. \\

Bei Ansatz 2 kann man aufgrund von mangelnden Daten keine Bewertung der Berechnungen durchf\"uhren. Weil es sich um berchnete relative H\"aufigkeiten
handelt, macht es keinen Sinn in einen Testdatensatz und Trainingsdatensatz zu unterscheiden, welche zuf\"allig gew\"ahlt werden. Es werden die relativen
H\"aufigkeiten berechnet, die man an einem Testzeitraum in der Zukunft bewerten m\"usste. \\

Bei Ansatz 3 kann wie oben angef\"uhrt die gesch\"atzte Anzahl an pr\"ufungsaktiven Studierenden mit der tats\"achlichen Anzahl verglichen werden. \\

Schlussendlich ist es das Ziel, dass die gesch\"atzte Anzahl an pr\"ufungsaktiven Studierenden im Jahr $t = 3$ m\"oglichst nahe an der tats\"achlichen Anzahl
liegt.



\subsection{Ans\"atze f\"ur P2}

Man hat bei P2 nicht ausreichend Daten, um die Studienbeginner in den kommenden Jahren  vorherzusagen. Das bedeutet auch, dass es innerhalb der vorhandenen
Daten nicht m\"oglich ist einen Testdatensatz zu bilden, an dem die beiden Ans\"atze f\"ur P2 verglichen werden k\"onnen. Bei beiden Ans\"atzen h\"angt
sehr viel von der Sch\"atzung der Anzahl der zuk\"unftigen Studienbeginner ab, f\"ur die man mehr Kalenderjahre im Datensatz ben\"otigen w\"urde, als
man zur Verf\"ugung hat. \\

F\"ur Ansatz 1 wird \"uberpr\"uft, ob es f\"ur 2 aus den 5 vorhandenen Kalenderjahren sinnvoll gewesen w\"are, diesen Ansatz zu w\"ahlen, oder ob sich die
Merkmalskombinationen stark von jenen im Vorjahr unterscheiden. Daf\"ur wird die Anzahl, welche bei einer tats\"achlichen Anwendung gesch\"atzt werden muss,
als gegeben angenommen. Anschlie{\ss}end vergleicht man, wie stark sich eine Sch\"atzung anhand der tats\"achlichen Merkmalskombination mit einer Sch\"atzung anhand von
Merkmalskombinationen aus dem Vorjahr unterscheiden. Zus\"atzlich werden diese beiden Werte mit der tats\"achlichen Anzahl an pr\"ufungsaktiven Neuinskripierenden
in diesem Kalenderjahr verglichen.\\

Weil Ansatz 2 noch mehr auf der Sch\"atzung der Anzahl der neuinskripierenden Studierenden aufbaut, kann man diesen Ansatz mit den vorhandenen Daten nicht bewerten oder legitimieren.
Zus\"atzlich zur Sch\"atzung der Anzahl der Studierenden wird die Sch\"atzung der Pr\"ufungsaktivit\"at so wie in P1 durchgef\"uhrt. Hierf\"ur kann man die
Ergebnisse aus P1 als einen Richtwert nehmen.