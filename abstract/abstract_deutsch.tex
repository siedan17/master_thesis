\documentclass[a4paper, german, 11pt]{article}
% fuer PDF A Format
\usepackage[a-2b]{pdfx}

%opening
\title{Zusammenfassung}
\author{Daniel Siemmeister}
\date{Graz, 2022}

\begin{document}

\maketitle
\thispagestyle{empty}

Das Ziel dieser Arbeit ist, ausgehend von Daten von Studierenden, zu sch\"atzen, wie viele
Studierende in 3 Jahren pr\"ufungsaktiv sein werden. Man will hierbei nur die Anzahl der pr\"ufungsaktiven Studierenden vorhersagen,
und nicht jeden einzelnen Studierenden individuell klassifizieren. \\

Um das Ziel der Arbeit zu erreichen, werden unterschiedliche Ans\"atze erprobt, welche auf verschieden Arten versuchen die Problemstellung zu l\"osen.
Die Problemstellung wird in zwei kleinere Unterproblemstellungen unterteilt. Die erste der beiden besch\"aftig sich mit der Vorhersage der pr\"ufungsaktiven Studierenden,
die bereits inskripiert sind. Die zweite Unterproblemstellung besch\"aftigt sich mit der Vorhersage von zuk\"unftigen Studierenden, die in den darauffolgenden beiden Jahren inskripieren werden.
In den meisten Ans\"atzen werden Machine Learning Modelle unterschiedlicher Architektur erprobt und hinsichtlich ihrer Vorhersagef\"ahigkeit
verglichen. \\

Die entscheidenden Einsichten der Arbeit sind:
\begin{itemize}
    \item Verst\"andnis der Problemstellung. Das hei{\ss}t, es ist nur die Anzahl der pr\"ufungsaktiven Studierenden gefragt, und man muss nicht jede Person exakt klassifizieren.
    \item Machine Learning Pr\"adiktoren klassifizieren anhand von gesch\"atzten Wahrscheinlichkeiten. Anstatt eines Schwellwertes zu klassifizieren,
          kann man diese Wahrscheinlichkeiten auch ohne Klassifizierung verwenden.
    \item Unterschiedliche Machine Learning Algorithmen liefern \"ahnliche Resultate. Die Komplexit\"at der Algorithmen ist nicht entscheidend.
    \item Anhand von gesch\"atzten Wahrscheinlichkeiten kann man die erwartete Anzahl an pr\"ufungsaktiven Studierenden passabel vorhersagen.
\end{itemize}

Auch wenn die Vorhersagen gute Ergebnisse liefern, muss beachtet werden, dass man mehr Daten ben\"otigt, um alle Ans\"atze hinreichend zu erproben.


\end{document}